\documentclass{article}
\usepackage{CJKutf8}
\usepackage{url}
\usepackage[colorlinks=true]{hyperref}
\usepackage{fancyhdr}
\usepackage{enumerate}
\usepackage{color,xcolor}
\usepackage{listings}
\lstset{
	basicstyle=\footnotesize, % size of fonts used for the code或改成\small\monaco稍大
	numbers=left,                        % 设置行号
	numberstyle=\tiny,            % 设置行号字体大小
	columns=fullflexible,
	breaklines=true,                 % automatic line breaking only at whitespace
	captionpos=b,                    % sets the caption-position to bottom
	tabsize=4,                       % 把tab扩展为4个空格,默认是8个太长
	commentstyle=\color{mygreen},    % 设置注释颜色
	escapeinside={\%*}{*)},          % if you want to add LaTeX within your code
	keywordstyle=\color{blue},       % 设置keyword颜色
	stringstyle=\color{mymauve},     % string literal style
	frame=none,                                          % 不显示背景边框
	backgroundcolor=\color[RGB]{245,245,244},            % 设定背景颜色
	rulesepcolor=\color{red!20!green!20!blue!20},
	language=c++,
}

% Math
\newcommand{\keyframe}{k-\Delta{k}}
\newcommand{\sd}[1]{\mathbf{#1}} % spatial domain
\newcommand{\fd}[1]{\hat{\mathbf{#1}}} % Fourier domin
\newcommand{\set}[2][n]{#2_1,#2_2,\dots,#2_#1}

% Ranking
\newcommand{\NoOne}[1]{\textcolor{red}{#1}}
\newcommand{\NoTwo}[1]{\textcolor{green}{#1}}
\newcommand{\NoThree}[1]{\textcolor{blue}{#1}}

% Remark
\newcounter{RNum}
\renewcommand{\theRNum}{\arabic{RNum}}
\newcommand{\Remark}{\noindent\textbf{Remark}~\refstepcounter{RNum}\textbf{\theRNum}: }

\renewcommand{\abstractname}{摘要}

%\fancypagestyle{plain}
%\fancyhf{} % 清空当前设置

\title{\LaTeX 测试手册}
\date{}
\author{Fuling~Lin\thanks{e-mail: linfuling5@tongji.edu.cn} ,
	Yujie~He\thanks{Hompage: \url{https://yujie-he.github.io/}, e-mail: 1551862@tongji.edu.cn}
}

\begin{document}
\begin{CJK}{UTF8}{gkai}


\maketitle
\begin{abstract}
随手用来测试的\LaTeX 文件。
\end{abstract}

\section{命令内容}
%使用如下命令免去写作繁琐输入:

%{\centering
%\texttt{$\backslash$newcommand\{新命令\}[参数数量][默认值]\{定义内容\}}。\par
%}

%要求如下:
%\begin{itemize}
%	\item 新命令:符合命令构成规则,不能与系统和已调用宏包命令重名。
%	\item 参数数量:可选,用于指定该命令具有参数个数,默认为0,即无参数。
%	\item 默认值:可选,用于设定第一个参数的默认值,如果定义时给出默认值,表命令第一个参数可选,新命令最多只能有一个可选参数,而且必须是第一个参数。
%	\item 命令内容:涉及某个参数时用符号\#n表示,如\#1 \#2。
%	\item 新定义命令参数不得含有抄录命令$\backslash$verb和抄录环境verbatim以及相关命令和环境。
%\end{itemize}

\subsection{定义数学符号}

定义如下:
\begin{lstlisting}
\newcommand{\keyframe}{k-\Delta{k}}
\newcommand{\sd}[1]{\mathbf{#1}} % spatial domain
\newcommand{\fd}[1]{\hat{\mathbf{#1}}} % Fourier domin
\newcommand{\set}[2][n]{#2_1,#2_2,\dots,#2_#1}
\end{lstlisting}

输入:
\begin{lstlisting}
$\sd{w} \rightarrow \fd{w}$,
$\fd{w}^{\keyframe}$,
$[\set[n]{\fd{w}}]$,
$[\set[m]{\sd{x}}]$
\end{lstlisting}

编译效果如下:

$\sd{w} \rightarrow \fd{w}$,
$\fd{w}^{\keyframe}$,
$[\set[n]{\fd{w}}]$,
$[\set[m]{\sd{x}}]$.

\subsection{定义带颜色内容}

定义如下:
\begin{lstlisting}
\newcommand{\NoOne}[1]{\textcolor{red}{#1}}
\newcommand{\NoTwo}[1]{\textcolor{green}{#1}}
\newcommand{\NoThree}[1]{\textcolor{blue}{#1}}
\end{lstlisting}

输入:
\begin{lstlisting}
Xiaoming gets \NoOne{100}, 
Xiaoxiong gets \NoTwo{98}, 
Xiaoli gets only \NoThree{60}.
\end{lstlisting}

编译效果如下:

Xiaoming gets \NoOne{100}, Xiaoxiong gets \NoTwo{98}, Xiaoli gets only \NoThree{60}.

\subsection{自带编号}

定义如下:
\begin{lstlisting}
\newcounter{RNum}
\renewcommand{\theRNum}{\arabic{RNum}}
\newcommand{\Remark}{\noindent\textbf{Remark}
	~\refstepcounter{RNum}\textbf{\theRNum}: }
\end{lstlisting}

输入:
\begin{lstlisting}
\Remark Point 1\\
\Remark Point 2\\
\Remark Point 3
\end{lstlisting}

编译效果如下:

\Remark Point 1\\
\Remark Point 2\\
\Remark Point 3

\section{参考资料}
\begin{enumerate}[1)]
	\item \href{https://www.jianshu.com/p/9d53cc6a64b8}{\LaTeX 计数器}
	\item \href{https://www.jianshu.com/p/a69116a7dbe1}{\LaTeX 自定义命令}
\end{enumerate}


\end{CJK}
\end{document}
